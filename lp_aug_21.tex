\begin{tabularx}{\textwidth}{Y}
  {\large University of Evansville Lesson Plan Format } \\
  \arrayrulecolor{blue} \hline \\
\end{tabularx}


\arrayrulecolor{black} 
\begin{tabularx}{\textwidth}{|X|X|}
  \hline 
  \textcolor{blue}{Name:} Randall Helzerman         &   \textcolor{blue}{Student ID Number:} 0128861 \\
  \hline 
  \textcolor{blue}{Course Number:} EDUC-497-03 2025FA &   \textcolor{blue}{Instructor Name:} Dr. Laura Watkins\\
  \hline 
\end{tabularx}
\arrayrulecolor{black} 

\vskip 10pt

\begin{tabularx}{\textwidth}{Y}
  {\bf Lesson Overview} \\
\end{tabularx}
\arrayrulecolor{black}


\arrayrulecolor{black} 
\begin{tabularx}{\textwidth}{|X|X|}
  \hline 
  \textbf{Lesson Title: Applying Rule of Exponents} \\
  
  \hline 
  \textbf{Subject:} Math\\
  
  \hline 
      {
        \begin{tabularx}{\textwidth}{X|X}
          \hskip -6pt
          \textbf{Duration of Lesson:} 80 min & \textbf{Grade Level(s)/Course:} 8th grade\\
        \end{tabularx}
      } \\
      
      \hline
      
      \textbf{Lesson Description:}  \\
        
        \hline
        
        \textbf{Standards and/or Indicator(s):}  \\
        \textbf{Indiana Standard number:} 8.NS.3 \\
        \textbf{Text:} Given a numeric expression with common rational
        number bases and integer exponents, apply the properties of
        exponents to generate equivalent expressions.  \\
        \hline
        
        \textbf{Learning Objective(s)/Target:} {\tiny What do I want students to learn and be able to do at the end of the lesson? (I can statements)}
               {\begin{enumerate}
                 \item I can identify which rule of exponents is needed to simplify.
               \end{enumerate}} \\
               \hline
               
               \textbf{Lesson Resources/Technology: }  Cards numbered 1 thorugh 14 to mark ``stations'' around the room, and 12 cards with problems involving exponentials.\\
               \hline
               
               \textbf{Key Vocabulary:} {\tiny List all key vocabulary words that will be taught to help students understand the concepts in the lesson.} \\
               \hline
\end{tabularx}
\arrayrulecolor{black}

\pagebreak

\begin{tabularx}{\textwidth}{|p{0.5in}|X|}
  \hline
  \centerline{\textbf{\large Time}} &  \textbf{\large Instructional Sequence } \\
  \hline
  \textbf{10-20min} &  \textbf{\em Introduction/Anticipatory Set:}  Six problems involving eponentials were written on the board.   Then we proceed in 2 stages:
         {\begin{enumerate}
           \item For all 6 problems, students first write down the rule needed to simplify the expression.  We review the answers.
           \item Then the students solve the problems.  We review the answers.
         \end{enumerate}}\\
         \hline
         \textbf{10-20 min} &  \textbf{\em Demonstrate, Build, Apply Knowledge} Mrs. Toellle had the students to to page 81/82 in the workbook, which was where the students should have written down the various rules for \\
  \hline
  \textbf{N/A} &  \textbf{\em Depth of Knowledge Questions:} N/A \\
  \hline
  \textbf{10-20 min} &  \textbf{\em Guided Practice:} Worked through a study guide for the test to be given the next day. \\
  \hline
  \textbf{10-20 min} &  \textbf{\em Independent Practice:} An Edulastic problem set was given. \\
  \hline
  \textbf{1 min} &  \textbf{\em Assessment:} Edulastic lets the students check their answers 1 time, and all the results are wrappted up to give a read on how the students aer doing. \\
  \hline
  \textbf{10-20 min} &  \textbf{\em Wrap Up/Closing Activity/Reflection:} We did a ``math treasure hunt.''   Around the room cards with numbers from 1-14 were posted, which marked stations around the room.  At each station, there was a problem involving exponentials.  Underneith that was printed the answer to the problem at{\em a different} station around the room.   \\
  \hline
\end{tabularx}

\vskip 6pt

\begin{small}
\begin{tabularx}{\linewidth}{|p{2.1in}|X|}
  \hline
  \textbf{Differentiation/Accommodations and/or Modifications: } & \\
  \hline
  \textbf{Culturally Responsive Teaching/ Diversity and Inclusion: } & \\
  \hline
\end{tabularx}

\vskip 6pt

\begin{tabularx}{\linewidth}{|X|}
  \hline
  \textbf{Self-Reflection:} \\
  \textbf{My Teaching:} 
  \begin{enumerate}
  \item The big idea from the previous day as a metaphor, viz, that the problems were locks, and the rule of exponents were the keys to solve them.  In retrospect, this would have been more effective if the 
  \item .
  \end{enumerate} \\
  
  \textbf{The Students:}
  \begin{enumerate}
  \item .
  \item .
  \end{enumerate} \\
  
  \textbf{The Lesson:}
  \begin{enumerate}
  \item .
  \item .
  \end{enumerate} \\
  
  \hline
\end{tabularx}
\end{small}
