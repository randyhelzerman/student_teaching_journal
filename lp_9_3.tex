\begin{tabularx}{\textwidth}{Y}
  {\large University of Evansville Lesson Plan Format } 9/3/2025 TO 9/4/2025\\
  \arrayrulecolor{blue} \hline \\
\end{tabularx}


\arrayrulecolor{black} 
\begin{tabularx}{\textwidth}{|X|X|}
  \hline 
  \textcolor{blue}{Name:} Randall Helzerman
  &
  \textcolor{blue}{Student ID Number:} 0128861 \\
  \hline
  
  \textcolor{blue}{Course Number:} EDUC-497-03 2025FA
  &
  \textcolor{blue}{Instructor Name:} Dr. Laura Watkins\\
  \hline
  
\end{tabularx}
\arrayrulecolor{black} 

\vskip 10pt
  
\begin{tabularx}{\textwidth}{Y}
  {\bf Lesson Overview} \\
\end{tabularx}
\arrayrulecolor{black}


\arrayrulecolor{black} 
\begin{tabularx}{\textwidth}{|X|X|}
  \hline
  
  \textbf{Lesson Title:} \\
  \hline
  
  \textbf{Subject:} Math\\
  \hline
  
  {\begin{tabularx}{\textwidth}{X|X}
      \hskip -6pt
      \textbf{Duration of Lesson:} & \textbf{Grade Level(s)/Course:}
      8th grade\\
  \end{tabularx}} \\
  \hline
  
  \textbf{Lesson Description: {\tiny (Describe the primary nature
      e.g. hands-on, direct instruction, inquiry, project based
      etc. of the lesson)}} \\
  \hline
  
  \textbf{Standards and/or Indicator(s):} {\tiny Cut and paste from
    IDOE website here. Feel free to replicate the descriptors listed
    below to include more standards in this section.} \\
  
  \textbf{Indiana Standard number:} \\
  
  \textbf{Text:} \\
  
  \hline
  
  \textbf{Learning Objective(s)/Target:} {
    \begin{enumerate}
    \item I can solve simple equations using squares and cubes.
    \item I know how to use the pythagorean theorem to find the length
      of the hypotenuse.
    \end{enumerate}
  }   \\
  \hline
  
  \textbf{Lesson Resources/Technology: } Promethean board \\
  \hline
  
  \textbf{Key Vocabulary:} \\ square, square root, cube, cube root,
  consecutive, hypotenuse, pythagorean theorem, operator, inverse
  operator. \\
  \hline
  
\end{tabularx}
\arrayrulecolor{black}

\pagebreak

\begin{tabularx}{\textwidth}{|p{0.5in}|X|}
  \hline
  
  \centerline{\textbf{\large Time}} &  \textbf{\large Instructional Sequence } \\
  
  \hline
  
  \textbf{} & \textbf{\em Introduction/Anticipatory Set:} Write the following simple equtions on the whiteboard:
  \[
    \begin{array}{c c c}
      c^{2}=15  & c^{2}=35  & c^{2}=8 \\
      c^{2}=27  & c^{2}=12  & c^{2}=17 \\
    \end{array} \]

   and ask the students to solve.  Then go over their answers. \\
  \hline
  
  \textbf{} & \textbf{\em Demonstrate, Build, Apply Knowledge}  N/A\\
  \hline
  
  \textbf{} & \textbf{\em Depth of Knowledge Questions:} N/A \\
  \hline
  
  \textbf{} & \textbf{\em Guided Practice:}   Worked a few problems on the study guide on the board.\\
  \hline
  
  \textbf{} & \textbf{\em Independent Practice:}  Study guide for test tomorrow. \\
  \hline
  
  \textbf{} & \textbf{\em Assessment:}  Test tomorrow \\
  \hline
  
  \textbf{} & \textbf{\em Wrap Up/Closing Activity/Reflection:} N/A \\
  \hline 
\end{tabularx}

\vskip 6pt

\begin{small}
  \begin{tabularx}{\linewidth}{|p{2.1in}|X|}
    \hline
    \textbf{Differentiation/Accommodations and/or Modifications: } & N/A\\
    \hline
    \textbf{Culturally Responsive Teaching/ Diversity and Inclusion: } & N/A\\
    \hline
  \end{tabularx}
  
  \vskip 6pt

  \begin{tabularx}{\linewidth}{|X|}
    \hline
    \textbf{Self-Reflection:} \\
    \textbf{My Teaching:} 
    \begin{enumerate}
    \item Made a critical mistake: emphasized that we want the answer to
      be consecutive whole numbers, but then explained how to solve the
      problems by advertint to perfect squares, which are not
      consecutive.  This confused a lot of students.  Fortunatey I fixed
      in in the subsequent presentations.
    \end{enumerate} \\
    
    \textbf{The Students:}
    \begin{enumerate}
    \item The students have a rather limited vocabulary, and we have to
      be sensitive to that.  The word ``consecutive'' was new to them,
      and it took a few days for them to wrap their heads around it.
    \end{enumerate} \\
2    
    \textbf{The Lesson:}
           \begin{enumerate}
             \item We are trying to adhere to the prescribed lessons as much as
               possible, but this particular unit did not seem to have very
               effective structure.  To the students, it just seem like a grab-bag of ad-hoc techniques.
             \item They are not in algebra class, but yet we are teaching them
               bits and pieces of algebra in order for them to be able to solve
               the problems.   It seems like presenting the rules for, say, canceling a square root with a square, would be easier to understand as part of a larger system of canceling addition with subtraction, etc.  
           \end{enumerate} \\
           
           \hline
\end{tabularx}
\end{small}
