\begin{tabularx}{\textwidth}{Y}
  {\large University of Evansville Lesson Plan Format (8/26/2025} \\
  \arrayrulecolor{blue} \hline \\
\end{tabularx}


\arrayrulecolor{black} 
\begin{tabularx}{\textwidth}{|X|X|}
  \hline 
  \textcolor{blue}{Name:} Randall Helzerman           &   \textcolor{blue}{Student ID Number:} 0128861 \\
  \hline 
  \textcolor{blue}{Course Number:} EDUC-497-03 2025FA &   \textcolor{blue}{Instructor Name:} Dr. Laura Watkins\\
  \hline 
\end{tabularx}
\arrayrulecolor{black} 

\vskip 10pt

\begin{tabularx}{\textwidth}{Y}
  {\bf Lesson Overview} \\
\end{tabularx}
\arrayrulecolor{black}


\arrayrulecolor{black}
\begin{tabularx}{\textwidth}{|X|X|}
  \hline 
  \textbf{Lesson Title:} Squares and Cubes and Repeating and Non-repeating
  Decimals\\
  
  \hline 
  \textbf{Subject:} Math\\
  
  \hline 
      {
        \begin{tabularx}{\textwidth}{X|X}
          \hskip -6pt
          
          \textbf{Duration of Lesson:} 80 min
          &
          \textbf{Grade Level(s)/Course:} 8th grade \\
        \end{tabularx}
      } \\
      
      \hline
      
      \textbf{Lesson Description:} In this lesson we get the students
      familier with squares and cubes of small and medium-sized
      integers.  \\
      
      \hline
      
      \textbf{Standards and/or Indicator(s):} \\
      \textbf{Indiana Standard number:} 8.NS.3 \\
      \textbf{Text:} Given a numeric expression with common rational
      number bases and integer exponents, apply the properties of
      exponents to generate equivalent expressions.  \\
      \hline
      
      \textbf{Indiana Standard number:} 8.NS.4 \\
      \textbf{Text:} Use square root symbols to represent solutions to
      equations of the form $x^2 = p$, where $p$ is a positive
      rational number. \\
      
      \hline
      
      \textbf{Learning Objective(s)/Target:} {\tiny What do I want
        students to learn and be able to do at the end of the lesson?
        (I can statements)}
      
      {\begin{enumerate}
        \item I can tell when a fraction has a repeating or
          non-repeating decimal.
        \item I can tell when a number is a squared number.
        \item I can tell when a number is a cubed number.
      \end{enumerate}} \\
      \hline
      
      \textbf{Lesson Resources/Technology: } {\tiny (What
        technology will I use? What technology will students
        use for this lesson? What resources will you and/or the
        students need for the lesson – books, mentor texts,
        etc.)} \\ \hline
      
      \textbf{Key Vocabulary:} {\tiny List all key vocabulary
        words that will be taught to help students understand
        the concepts in the lesson.} \\ \hline
\end{tabularx}
\arrayrulecolor{black}

\pagebreak

\begin{tabularx}{\textwidth}{|p{0.5in}|X|}
  \hline
  
  \centerline{\textbf{\large Time}}
  &
  \textbf{\large Instructional Sequence} \\
  \hline
  
  \textbf{10 min} & \textbf{\em Introduction/Anticipatory Set:}  Six
  problems put on promeathean board to drill properties of exponents,
  follwed by review. \\
  \hline
  
  \textbf{} & \textbf{\em Demonstrate, Build, Apply Knowledge}  Work
  through class notes in workbook Lesson 17 on solving eqations with
  squares and cubes.  \\
  \hline
  
  \textbf{} & \textbf{\em Depth of Knowledge Questions:}  N/A \\
  \hline
  
  \textbf{} & \textbf{\em Guided Practice:} 
  Assignment of in-class exercises from the workbook, Lesson 17.  \\
  \hline
  
  \textbf{} & \textbf{\em Independent Practice:} {\tiny (Individual
    practice, learning centers, reading, composing writing; “You do”)} \\
  \hline
  
  \textbf{} &  \textbf{\em Assessment:} Exit ticket from workbook\\
  \hline

  \textbf{} &  \textbf{\em Wrap Up/Closing Activity/Reflection:} N/A \\
  \hline
  
\end{tabularx}

\vskip 6pt

\begin{small}
  \begin{tabularx}{\linewidth}{|p{2.1in}|X|}
    \hline
    \textbf{Differentiation/Accommodations and/or Modifications: } & \\
    \hline
    \textbf{Culturally Responsive Teaching/ Diversity and Inclusion: } & N/A\\
    \hline
  \end{tabularx}
  
  \vskip 6pt
  
  \begin{tabularx}{\linewidth}{|X|}
    \hline
    \textbf{Self-Reflection:} \\
    \textbf{My Teaching:} 
    \begin{enumerate}
    \item Something is missing.  Even after doing the Fluency
      exercise, some students are still not tracking.  They can't do
      the exercizes by themselves and they can't verbally explain what
      is going on even in the most general terms.
      
    \item More effort made to keept he students engaged during the
      lectures.  They should be writing in their books taking notes, and
      a lot of them need to be called back.
    \end{enumerate} \\
    
    \textbf{The Students:}
    \begin{enumerate}
      
    \item 5 or 6 students in every session have already learned the
      material because they have worked ahead in their workbooks and
      IXLs.
    \item Some students are getting left behind.
    \item Lots of chit chat at inapropos times.
    \end{enumerate} \\
    
    \textbf{The Lesson:}
    \begin{enumerate}
    \item The lesson appears to have words in it which some students
      don't know.
    \end{enumerate} \\
    
    \hline
  \end{tabularx}
\end{small}
