\begin{tabularx}{\textwidth}{Y}
  {\large University of Evansville Lesson Plan Format } \\
  \arrayrulecolor{blue} \hline \\
\end{tabularx}


\arrayrulecolor{black} 
\begin{tabularx}{\textwidth}{|X|X|}
  \hline 
  \textcolor{blue}{Name:}          &   \textcolor{blue}{Student ID Number:} \\
  \hline 
  \textcolor{blue}{Course Number:} &   \textcolor{blue}{Instructor Name:} \\
  \hline 
\end{tabularx}
\arrayrulecolor{black} 

\vskip 10pt
  
\begin{tabularx}{\textwidth}{Y}
  {\bf Lesson Overview} \\
\end{tabularx}
\arrayrulecolor{black}


\arrayrulecolor{black} 
\begin{tabularx}{\textwidth}{|X|X|}
  \hline 
  \textbf{Lesson Title:} \\
  
  \hline 
  \textbf{Subject:} \\
  
  \hline 
      {
        \begin{tabularx}{\textwidth}{X|X}
          \hskip -6pt
          \textbf{Duration of Lesson:} & \textbf{Grade Level(s)/Course:} \\
        \end{tabularx}
      } \\
      
      \hline
      
      \textbf{Lesson Description: {\tiny (Describe the primary nature e.g. hands-on, direct instruction, inquiry, project based etc. of the lesson)}} \\
        
        \hline
        
        \textbf{Standards and/or Indicator(s):} {\tiny Cut and paste from IDOE website here. Feel free to replicate the descriptors listed below to include more standards in this section.} \\
        \textbf{Indiana Standard number:} \\
        \textbf{Text:} \\
        \hline
        
        \textbf{Learning Objective(s)/Target:} {\tiny What do I want students to learn and be able to do at the end of the lesson? (I can statements)}
               {\begin{enumerate}
                 \item Lorum
                 \item ipsum
               \end{enumerate}} \\
               \hline
               
               \textbf{Lesson Resources/Technology: } {\tiny (What technology will I use? What technology will students use for this lesson? What resources will you and/or the students need for the lesson – books, mentor texts, etc.)} \\
               \hline
        
               \textbf{Key Vocabulary:} {\tiny List all key vocabulary words that will be taught to help students understand the concepts in the lesson.} \\
               \hline
\end{tabularx}
\arrayrulecolor{black}

\pagebreak

\begin{tabularx}{\textwidth}{|p{0.5in}|X|}
  \hline
  \centerline{\textbf{\large Time}} &  \textbf{\large Instructional Sequence } \\
  \hline
  \textbf{} &  \textbf{\em Introduction/Anticipatory Set:} {\tiny What meaningful activity will engate students, activate prior knowledge, and prepare students for learning objectives} \\
  \hline
  \textbf{} &  \textbf{\em Demonstrate, Build, Apply Knowledge} {\tiny What meaningful activity will engate students, activate prior knowledge, and prepare students for learning objectives} \\
  \hline
  \textbf{} &  \textbf{\em Depth of Knowledge Questions:} {\tiny Essential questions to extend higher level thinking)
} \\
  \hline
  \textbf{} &  \textbf{\em Guided Practice:} {\tiny (Check for understanding prior to independent work; “We do”)} \\
  \hline
  \textbf{} &  \textbf{\em Independent Practice:} {\tiny (Individual practice, learning centers, reading, composing writing; “You do”)} \\
  \hline
  \textbf{} &  \textbf{\em Assessment:} {\tiny (Evaluate level of student understanding. How will I know if students have achieved today’s learning target?)} \\
  \hline
  \textbf{} &  \textbf{\em Wrap Up/Closing Activity/Reflection:} {\tiny (How will I reinforce/revisit the learning objective? Opportunity for formative assessment. Students reflect on evidence of learning – writing, reading, math target. How will we build on this learning?)} \\
  \hline
\end{tabularx}

\vskip 6pt

\begin{small}
\begin{tabularx}{\linewidth}{|p{2.1in}|X|}
  \hline
  \textbf{Differentiation/Accommodations and/or Modifications: } & \\
  \hline
  \textbf{Culturally Responsive Teaching/ Diversity and Inclusion: } & \\
  \hline
\end{tabularx}

\vskip 6pt

\begin{tabularx}{\linewidth}{|X|}
  \hline
  \textbf{Self-Reflection:} \\
  \textbf{My Teaching:} 
  \begin{enumerate}
  \item .
  \item .
  \end{enumerate} \\
  
  \textbf{The Students:}
  \begin{enumerate}
  \item .
  \item .
  \end{enumerate} \\
  
  \textbf{The Lesson:}
  \begin{enumerate}
  \item .
  \item .
  \end{enumerate} \\
  
  \hline
\end{tabularx}
\end{small}
