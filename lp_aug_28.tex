\begin{tabularx}{\textwidth}{Y}
  {\large University of Evansville Lesson Plan Format 8/28/2025} \\
  \arrayrulecolor{blue} \hline \\
\end{tabularx}


\arrayrulecolor{black} 
\begin{tabularx}{\textwidth}{|X|X|}
  \hline 
  \textcolor{blue}{Name:} Randall Helzerman         &   \textcolor{blue}{Student ID Number:} 0128861 \\
  \hline 
  \textcolor{blue}{Course Number:} EDUC-497-03 2025FA &   \textcolor{blue}{Instructor Name:} Dr. Laura Watkins\\
  \hline 
\end{tabularx}
\arrayrulecolor{black} 

\vskip 10pt
  
\begin{tabularx}{\textwidth}{Y}
  {\bf Lesson Overview} \\
\end{tabularx}
\arrayrulecolor{black}


\arrayrulecolor{black} 
\begin{tabularx}{\textwidth}{|X|X|}
  \hline 
  \textbf{Lesson Title:} Squares and how to use them in the Pythagorean Theorem\\
  
  \hline 
  \textbf{Subject:} Math \\
  
  \hline 
      {
        \begin{tabularx}{\textwidth}{X|X}
          \hskip -6pt
          
          \textbf{Duration of Lesson:} 80 min & \textbf{Grade
            Level(s)/Course:} 8th grade\\
        \end{tabularx}
      } \\
      
      \hline
      
      \textbf{Lesson Description:} The students learn to recognize
      numbers which are perfect square, to ease their way into
      learning about and using the pythagorean theorem.  \\
      
      \hline
      
      \textbf{Standards and/or Indicator(s):} {\tiny Cut and paste
        from IDOE website here. Feel free to replicate the descriptors
        listed below to include more standards in this section.} \\
      
      \textbf{Indiana Standard number:} \\
      \textbf{Text:} \\
      \hline
      \textbf{Learning Objective(s)/Target:} {\tiny What do I want
        students to learn and be able to do at the end of the lesson?
        (I can statements)}
      {\begin{enumerate}
        \item I can say the squares of the first ten numbers from memory.
        \item 
      \end{enumerate}} \\
      \hline
      \textbf{Lesson Resources/Technology:} Approx. 40 3x5 cards with
      questionson one side and answers on the other for Quiz-Quiz
      Trade.  \\
      \hline
      \textbf{Key Vocabulary:} Hypoteneuse, Right Angle, Right
      triangle, Legs of a Right Triangle \\
      
      \hline
\end{tabularx}
\arrayrulecolor{black}

\pagebreak

\begin{tabularx}{\textwidth}{|p{0.5in}|X|}
  \hline
  
  \centerline{\textbf{\large Time} 80 min}
  &
  \textbf{\large Instructional Sequence } \\
  \hline
  
  \textbf{} &  \textbf{\em Introduction/Anticipatory Set:}  \\
  10 min & Write six simple problems involving squares and cubes on the board, and have students solve.  Discuss afterwords.\\
  \hline
  
  \textbf{}
  &
  \textbf{\em Demonstrate, Build, Apply Knowledge} We do a Quiz-Quiz
  Trade activity (see cards we used in figure \ref{qqt_root}) in order
  to give the students the opportunity \\
  
  \hline
  
  \textbf{} & \textbf{\em Depth of Knowledge Questions:} {\tiny
    Essential questions to extend higher level thinking)} \\
  \hline
  
  \textbf{20 min}
  &
  \textbf{\em Guided Practice:} Lots of cold-calling.  Random students were asked how to solve square and cube root problems.   Many did not know how yet, but were allowed to ``phone a friend'', i.e. they were able to choose a student who had their hand up.  That student had to explain to the first student, and that student had to explain to me!

  Sometimes this went even more levels...a friend calling a friend calling a friend....then they all had to explain it back to each othre along the chain.  Hopefully this helped them remember...
  \\
  \hline
  
  \textbf{} & \textbf{\em Independent Practice:} N/A\\
  \hline
  
  \textbf{} & \textbf{\em Assessment:} This one was pretty much only for student D, but we were very proud of him for being able to demonstrate how to find the cube root of 125 at the promethean board in front of the entire class.  \\
  \hline
  
  \textbf{} & \textbf{\em Wrap Up/Closing Activity/Reflection:} N/A \\
  \hline
  
\end{tabularx}

\vskip 6pt

\begin{small}
\begin{tabularx}{\linewidth}{|p{2.1in}|X|}
  \hline
  \textbf{Differentiation/Accommodations and/or Modifications: } & N/A\\
  \hline
  \textbf{Culturally Responsive Teaching/ Diversity and Inclusion: } & N/A\\
  \hline
\end{tabularx}

\vskip 6pt

\begin{tabularx}{\linewidth}{|X|}
  \hline
  \textbf{Self-Reflection:} \\
  \textbf{My Teaching:} 
  \begin{enumerate}
  \item Cold calling is always something I had trouble with.  But I am getting more comfortable with it.
  \end{enumerate} \\
  
  \textbf{The Students:}
  \begin{enumerate}
  \item Some students get it immediatly and are bored.  I've got to crack that nut.
  \end{enumerate} \\
  
  \textbf{The Lesson:}
  \begin{enumerate}
  \item Nothing special to say...
  \end{enumerate} \\
  
  \hline
\end{tabularx}
\end{small}
