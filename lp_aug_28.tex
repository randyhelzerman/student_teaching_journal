\begin{tabularx}{\textwidth}{Y}
  {\large University of Evansville Lesson Plan Format } \\
  \arrayrulecolor{blue} \hline \\
\end{tabularx}


\arrayrulecolor{black} 
\begin{tabularx}{\textwidth}{|X|X|}
  \hline 
  \textcolor{blue}{Name:} Randall Helzerman         &   \textcolor{blue}{Student ID Number:} 0128861 \\
  \hline 
  \textcolor{blue}{Course Number:} EDUC-497-03 2025FA &   \textcolor{blue}{Instructor Name:} Dr. Laura Watkins\\
  \hline 
\end{tabularx}
\arrayrulecolor{black} 

\vskip 10pt
  
\begin{tabularx}{\textwidth}{Y}
  {\bf Lesson Overview} \\
\end{tabularx}
\arrayrulecolor{black}


\arrayrulecolor{black} 
\begin{tabularx}{\textwidth}{|X|X|}
  \hline 
  \textbf{Lesson Title:} Squares and how to use them in the Pythagorean Theorem\\
  
  \hline 
  \textbf{Subject:} Math \\
  
  \hline 
      {
        \begin{tabularx}{\textwidth}{X|X}
          \hskip -6pt
          
          \textbf{Duration of Lesson:} 80 min & \textbf{Grade
            Level(s)/Course:} 8th grade\\
        \end{tabularx}
      } \\
      
      \hline
      
      \textbf{Lesson Description:} The students learn to recognize
      numbers which are perfect square, to ease their way into
      learning about and using the pythagorean theorem.  \\
      
      \hline
      
      \textbf{Standards and/or Indicator(s):} {\tiny Cut and paste
        from IDOE website here. Feel free to replicate the descriptors
        listed below to include more standards in this section.} \\
      
      \textbf{Indiana Standard number:} \\
      
      \textbf{Text:} \\
      
      \hline
      
      \textbf{Learning Objective(s)/Target:} {\tiny What do I want
        students to learn and be able to do at the end of the lesson?
        (I can statements)}
      
      {\begin{enumerate}
        \item I can say the squares of the first ten numbers from memory.
        \item 
      \end{enumerate}} \\
      \hline
      
      \textbf{Lesson Resources/Technology:} Approx. 40 3x5 cards with
      questionson one side and answers on the other for Quiz-Quiz
      Trade.  \\
      
      \hline
      
      \textbf{Key Vocabulary:} Hypoteneuse, Right Angle, Right
      triangle, Legs of a Right Triangle \\
      
      \hline
\end{tabularx}
\arrayrulecolor{black}

\pagebreak

\begin{tabularx}{\textwidth}{|p{0.5in}|X|}
  \hline

  \centerline{\textbf{\large Time}} &  \textbf{\large Instructional Sequence } \\
  \hline
  
  \textbf{} &  \textbf{\em Introduction/Anticipatory Set:}  \\
  Write six simple problems involving squares and cubes \\
  \hline
  
  \textbf{} & \textbf{\em Demonstrate, Build, Apply Knowledge} {\tiny
    What meaningful activity will engate students, activate prior
    knowledge, and prepare students for learning objectives} \\
  \hline
  
  \textbf{} & \textbf{\em Depth of Knowledge Questions:} {\tiny
    Essential questions to extend higher level thinking)} \\
  \hline

  \textbf{} & \textbf{\em Guided Practice:} {\tiny (Check for
    understanding prior to independent work; “We do”)} \\
  \hline
  
  \textbf{} & \textbf{\em Independent Practice:} {\tiny (Individual
    practice, learning centers, reading, composing writing; “You do”)} \\
  \hline
  
  \textbf{} & \textbf{\em Assessment:} {\tiny (Evaluate level of
    student understanding. How will I know if students have achieved
    today’s learning target?)} \\
  \hline
  
  \textbf{} & \textbf{\em Wrap Up/Closing Activity/Reflection:} {\tiny
    (How will I reinforce/revisit the learning objective? Opportunity
    for formative assessment. Students reflect on evidence of learning
    – writing, reading, math target. How will we build on this
    learning?)} \\
  \hline
  
\end{tabularx}

\vskip 6pt

\begin{small}
\begin{tabularx}{\linewidth}{|p{2.1in}|X|}
  \hline
  \textbf{Differentiation/Accommodations and/or Modifications: } & \\
  \hline
  \textbf{Culturally Responsive Teaching/ Diversity and Inclusion: } & \\
  \hline
\end{tabularx}

\vskip 6pt

\begin{tabularx}{\linewidth}{|X|}
  \hline
  \textbf{Self-Reflection:} \\
  \textbf{My Teaching:} 
  \begin{enumerate}
  \item .
  \item .
  \end{enumerate} \\
  
  \textbf{The Students:}
  \begin{enumerate}
  \item .
  \item .
  \end{enumerate} \\
  
  \textbf{The Lesson:}
  \begin{enumerate}
  \item .
  \item .
  \end{enumerate} \\
  
  \hline
\end{tabularx}
\end{small}
