\documentclass[11pt]{elegantbook}

\title{Lesson Plan Journal}
\subtitle{Mrs. Patricia Toelle's 8th-Grade Math Classes \\ at Perry Heights Middle School}

\author{Randall Helzerman}
\institute{University of Evansville}
\date{\today}
\version{0.1}
%\bioinfo{Bio}{Information}

%\logo{logo-blue.png}
\cover{cover.png}

% modify the color in the middle of titlepage
\definecolor{customcolor}{RGB}{32,178,170}
\colorlet{coverlinecolor}{customcolor}
\usepackage{cprotect}

\addbibresource[location=local]{reference.bib} % bib

\begin{document}

\maketitle

\frontmatter
\tableofcontents

\mainmatter

%\preface

\part{August}

\chapter{8/4/2025 through 8/9/2025}

This is a reflection on the first week of classes, being a Student
Teacher in Mrs. Patricia Toelle's 8th grade Math classes.  I am
rolling up the entire first week into a single reflection, because
what happened in the first 5 days divides more naturally into thematic
categories, not temporal categories.

\section*{The room and use of physical space}

Perry Heights Middle School is a very charming school, with a small
(440ish) number of students, and a correspondingly small physical
plant, and, alas, correspondingly small rooms.
ways--fitting all the necessary desks takes up almost all of the free

However, I think that perhaps these smaller rooms might feel more cozy
and reassuring to a middle school student, compared with the vast
cavernous rooms which Central High School had.  The rooms and hallways
are carpeted, which was a big surprise to me, as I've never been in a
school which didn't use industrial-grade linoleum.  However, it does
make the room somehow softer and warmer.  In addition, my aging ankles
appreciated standing on a softer surface all day!

\section*{Use of wall space}

The lion's share of the wall space is taken up by 4 whiteboards, which
are used for stations for activities.  One of the whiteboards has four
pencils stuck on, and next to them a space for a child to write their
name when they borrow a pencil. This is an excellent system, IMHO.

Nestled between is a series of posters which I hope will change the
lives of both the students and myself: posters which show various
slogans illustrating the growth mind set.  Thoughts become words, words
become actions, actions become habits.

\section*{Activities and Rituals}

To further show the growth mind set, Mrs. Toelle gave the children a
baggy of statements like ``I tried it and it didn't work'', and ``I'll
find another way to do it'', things like that.   She had the children
sort the statements as to whether they were characteristic of the growth
mind set or the fixed mind set.

Interestingly, the children were able to sort the statements quite
accurately.  Now that they know the growth mind set, lets hope they
practice it!

Side note: I'm trying to adopt a growth mind set myself w.r.t. parts of
teaching which have been hard for me, such as learning the student's
names.  There are about 75 students in our classes, and I've learned
about 60 names....not reaching the goal of learning them all by the
end of the first week, but frankly more names than I learned at either
Bosse or Central.

Mrs. Toelle has a ritual called the ``community circle'' which I think
was used very effectively.  The kids all gather in a circle, and she
asks a question designed to let us get to know more about ourselves.
Examples include ``what was the funniest thing you did this summer''
and ``what are you most dreading about the semester.''

She then gives students a few minutes to think.  When a student has an
answer, they give a thumbs up.  When a quorum of thumbs are up we pass
a ``magic marble'' around the circle, and the one holding the marble
is the only one allowed to speak.  I say a quorum and not every
student, because the student always has the right to just say
``pass.''  They do, however, have to say their name, which I appreciate
very much, as it aids in learning their names quite considerably.

I am not a superstitious man, but I am almost prepared to believe it
is a magic marble, because all Mrs. Toelle has to say is ``Respect the
Marble!!'' and all side chitter-chatter stops.   Astounding.

\section*{On the teaching of 8th graders}

I had some trepadations about whether I would be compatible with 8th
graders.  I never taught a freshman class, but I did hear other
teachers complaining about how ill-behaved the freshmen were.

While it is true, these children are not as mature as the juniors and
seniors I have taught before, they are unambiguously {\em children},
which means that sure they have more exuberance than upperclassmen do,
they also are more responsive to adult guidance.   

I'm delighted to announce that I absolutely love working with 8th graders.
I was not at all uncomfortable joining in the activities, yelling out
the slogans, etc.  These children have fortunately not had all of the joy
of life beaten out of them yet.  Even playing a game like ``four corners'',
just scurrying across the room--merely walking on the floor--was a
source of delight for them.

I certainly hope that I can help them mature intellectually and
morally, without squashing that thrill they have of just being alive.

\chapter{Monday, 8/11/2025}

\section*{Community Circle}

An interesting thing happened when we did the community circle with
the flowing question: ``What I need to feel safe in class''.  A few of
the students were brave enough to give answers like ``I need the
teacher to not make me feel stupid if I answer a question wrong.''
But most were very uncomfortable answering this question in the
standard community circle format.  Most of the students passed.

So Mrs. Toelle came up with an alternate plan: have the children write
down what they needed to feel safe anonymously, on sticky notes.
It was important that this was anonymous, as evinced by a student
who didn't hear at first that they were to be anonymous, but found out
halfway through his note.  He said, ``We don't write down our names?
That changes everything!''  Then he proceeded to erase what he set down
and write out what he really needed.

\section*{IXL}

The students are required to finish 20 minutes of IXL a week, which is
a computer-based series of questions, individualized for each student
based on their current mastery.  They are required to get an 80%
score.

Mrs. Toelle uses this as a way for students to demonstrate that they have
put in effort to learn the material as a pre-requisite for re-taking tests.

\section*{Workbook Assignment}

The students are then given the opportunity to work thorough exercises
in their workbook, in order to learn how to write very large numbers
and very small numbers in terms of powers of 10.  The workbook has a
handy graphic organizer which lets the students figure out whether, 
say, 0.009 has its first nonzero digit in the tenths, hundrethds, or
thousandths place.



\chapter{Tuesday, 8/12/2025}

\section*{Intro, warm up}

Mrs. Toelle had a few what she called ``fluency'' exercises, meant to
bridge from what the students hopefully remembered to what they needed
to learn today.

The first few problems displayed on the Promethean board were examples
of writing large numbers like 8,000,000 as a digit times 10 to some
power.

She then had them try to estimate the following two quantities:
\begin{enumerate}
\item How many different subway sandwiches can be made by combining
  ingredients?
\item How large is a single COVID virus?
\end{enumerate}

\noindent I noted with interest that the student's estimations were
off by at least two orders of magnitude in either direction.  But the
point wasn't to get the right answer, the point was whether you could
express your guess as a number.

\section*{Main subject matter}

The main lesson consisted of exercises and practice in converting
large numbers to and from exponential notation.  ``Exponential
notation'' is a variant of scientific notation, which, curiously, does
not use negative exponents (e.g. $10^{-1}$ for numbers less than 1.
Instead, they used reciprocals of powers of
10. (i.e. $\frac{1}{10^{2}}$.

Frankly, I have mixed feelings about this, as it is non-standard
notation.  But, it is not {\em incorrect} notation, and they are
careful not to call it ``scientific notation'' but rather
``exponential notation.'' And I cannot deny that it is a smoother
on-ramp.

\section*{Landing, evaluation}

One session (session 3/4) was on schedule enough to be able to take
what is termed an ``exit ticket.''  It is printed on a sheet in their
workbooks, which they are obliged to rip out and complete.

It consisted of about 4 problems which are meant to be very easy if
the student has understood the lessons.  How were these used? Stay
tuned for tomorrow's reflection.


\chapter{Wednesday, 8/13/2025}

\section*{Small Groups to review exit ticket}

The exit tickets from yesterday have already been graded, and the
children are who did not get $100\%$ have been categorized into small
groups for a more focused instruction.  The groups are divided up
between Mrs. Toelle and myself.  We then re-explain the concept, have
them drill on a few problems, and then correct their exit ticket.

\section*{IXL practice}

For the students who did get $100\%$, and for the students who have
corrected their exit tickets, we have them practicing on their IXL
problems.

\section*{Fluency and In-class practice}

After the the small groups, we do another fluency exercise for review,
followed by in-class practice from the workbook.

\section*{Exit ticket for classes 1/2 and 8/9}

Classes 1/2 and 8/9 had a chance to take the exit ticket which class
3/4 took the day before,



\chapter{Thursday, 8/14/2025}

\section*{Small Groups}

The exit tickets for classes 1/2 and 8/9 had been graded, so the children
were assigned to smaller groups as was done for class 3/4 the day before, and
given more focused tutelage in smaller groups, while the rest of the class
did their IXL assignments.

\section*{Fluency and independent practice}

Afterwards, the class did the fluency exercise, and then was given
time for in-class practice in their workbooks.  Afterwards, the answers
were checked.


\chapter{Friday 8/15/2025}


\section*{Observation of Honors Geometry Class}

Mrs. Toelle did me a solid and arranged for me to observe Mrs. Rice's
honors geometry class.  I had heard rumors of a mythically good
teacher, but what I saw with my own eyes utterly exceeded any report,
and indeed, and expectation that I could have had.

\section*{Description of classroom, and how it serves the learning process}

The classroom was bigger than the usual cozy Perry Heights classroom.
In contradistinction to what one would imagine that an 8th-grade
classroom was like (every square in of the walls decorated with cosy
pictures or encouraging slogans) the decor was spare, even stark.
Apparently, the honors students are already in possession of a growth
mindset, and do not need to be reminded of slogans to maintain it.

The desks were arranged in islands of 6 to 8 desks put together such
that the children all faced each other, and yet could all also see the
Promethean board at the front of the class.

This was for to provide an opportunity for groups of students to work
and struggle together.  Mrs. Rice strongly believes that it is
important for student to struggle together like this. This is in
contradistinction with most of my previous mentors who strongly
objected to any teaching technique which would cause the students to
feel perplexed for any length of time.

\section*{Teaching}

Interestingly, the students were not issued a workbook, or even a textbook.
Instead, they were issued graph-lined notebooks, like the notebooks
I used to use as a professional engineer.  The students were expected
to take notes in the notebook with a high enough fidelity that they
could refer back to it well enough not to need a textbook or workbook.
This is a very useful skill, which I used on the job every day.

The class opened with a tutorial of how to use a compass--compasses
having been previously passed out.  Now, I cannot resist commenting on
what passes for compasses these days.  As a card-carrying member of Gen X,
we did not have the safety concerns which are so prevalent today.
For us, a compass was something which could literally (not metaphorically or
hyperbolically, {\em literally}) be used as a lethal weapon, what with its
long, pointed, and very sharp end.  I.E., what students in every western
nation have been using as compasses since the days of Alexander the Great.

The instrument which was given to the kids had no such sharp pointing
parts.  It was basically a ruler with a pivot on one end which could
be held with a finger, and a sliding member which had a hole you could
put a pencil point into, and then you could pivot it around to draw a
circle.

I have very mixed feelings about this.  The whole game of geometry is
to see how far you can get with just a straight-edge and a compass.
On the one hand, I suppose that it is cool that the functionality of
both instruments can be combined into one, kind of.  On the other hand,
this does distance them from both the historical roots of geometry,
which is bad enough, but it also obscures the very different roles
which the two instruments play.   The ruler defines what a straight line
is, and the compass defines what it means to say that two lengths are
equal.

Nevertheless, I felt very privileged to be with these students at the
very moment they were learning how to use a compass.  It was a sacred
moment.

\section*{STRUGGLE!!}

Mrs. Rice proceeded to give them an assignment: use this compass to
bisect a line segment.  This is something which is easy to do once you
see the trick, but it is actually rather hard to come up with the
trick.

So she set the groups of students to chew on it, and so they did.
They were a classroom of 8th-graders, so naturally, there was plenty
of gigging and it was obvious the students were enjoying each other's
company.  But they did indeed struggle with the problem.  For a long
time!  Not every group could find the trick, but, when the trick was
revealed, there was a palpable sense of ``ah ha!!!!'' in the
classroom.  I never forgot how to bisect a line with a compass, and
I'm sure that these students--after having struggled for the
knowledge, will never forget either.

\section*{Techniques learned}

After the class, Mrs. Rice was kind enough to answer my questions and
share some new techniques.  One killer technique she showed me--not
for this geometry class, but for an algebra class-- was a poster-sized
graphical organizer she used to get students familiar with the
integers and how they are composed of prime factors.  The poster had
numbered boxes for numbers from 1 to 500.  She had the students put a
red sticker on every other number--those are the even numbers,
divisible by 2.  Then she had them put a green sticker on every third
box--those are the numbers divisible by 3.  Then a yellow sticker for
every 5th box--the numbers divisible by 5.

I thought this was an amazing technique, because it doesn't matter whether
a student is a verbal learner, a visual learner, or a procedural learner,
after doing this exercise, any student would be able to grasp on a very
visceral level the fact that all numbers are composed of prime factors.  And 8th-graders
are still young enough to love stickers, so I'm sure they had a really fun time
doing the exercise.

What's more,the poster would be useful for them when learning how to,
say, find greatest common factors or least common multiples, factoring
polynomials, etc etc.

\section*{Sad postscript}

I was wondering how Mrs. Rice could possibly get away with teaching
her students like this.  Certainly any time I tried to let my students
struggle, or I strayed off of the EVSC-mandated script, it was a point
of contention with my mentors.

I knew that EVSC has mandated what textbooks, workbooks, and even
lesson plans were to be used, and I mean used line-by-line, letter by
letter.  I suppose this gives a uniformity to instruction across
EVSC. But this drains the teacher of any discretion to use their own
unique talents adapted to their own unique group of students in their
own classroom.

When I asked Mrs. Rice how she could do this, she said, ``Well, I
suppose I could be written up for not following the EVSC lesson
plans.  If they fire me I'll just go do something else...''















\end{document}
