\begin{tabularx}{\textwidth}{Y}
  {\large University of Evansville Lesson Plan Format } \\
  \arrayrulecolor{blue} \hline \\
\end{tabularx}


\arrayrulecolor{black} 
\begin{tabularx}{\textwidth}{|X|X|}
  \hline 
  \textcolor{blue}{Name:} Randall Helzerman         &   \textcolor{blue}{Student ID Number:} 0128861 \\
  \hline 
  \textcolor{blue}{Course Number:} EDUC-497-03 2025FA &   \textcolor{blue}{Instructor Name:} Dr. Laura Watkins\\
  \hline 
\end{tabularx}
\arrayrulecolor{black} 

\vskip 10pt
  
\begin{tabularx}{\textwidth}{Y}
  {\bf Lesson Overview} \\
\end{tabularx}
\arrayrulecolor{black}


\arrayrulecolor{black} 
\begin{tabularx}{\textwidth}{|X|X|}
  \hline 
  \textbf{Lesson Title:} Final Prep for Test on Exponentials \\
  
  \hline 
  \textbf{Subject:} Math\\
  
  \hline 
      {
        \begin{tabularx}{\textwidth}{X|X}
          \hskip -6pt
          \textbf{Duration of Lesson:} 80 minutes & \textbf{Grade Level(s)/Course:} 8th grade\\
        \end{tabularx}
      } \\
      
      \hline
      
      \textbf{Lesson Description:} Final Prep for Test on Exponentials  \\
        
        \hline
        
        \textbf{Standards and/or Indicator(s):} \\
        \textbf{Indiana Standard number:} 8.NS.3 \\
        \textbf{Text:} Given a numeric expression with common rational
        number bases and integer exponents, apply the properties of
        exponents to generate equivalent expressions.  \\
        \hline
        
        \textbf{Learning Objective(s)/Target:}
             {\begin{enumerate}
               \item I can multiply terms containing exponents.
               \item I can identify coefficients in a term.
               \item I can identify bases in a term.
               \item I can identify exponents in a term.
               \item I can look at a term and see which rule of
                 exponents can be used to simplify it.
               \item I can solve simple equations using exponents.
             \end{enumerate}} \\
               \hline
               
               \textbf{Lesson Resources/Technology: }  3x5 index cards (30 count), with questions about exponentials printed out and taped to the front side, and corresponding answers printed out and taped to the back side.  \\
               \hline
               
               \textbf{Key Vocabulary:} Coefficient, Exponent, Base  \\
               \hline
\end{tabularx}
\arrayrulecolor{black}

\pagebreak

\begin{tabularx}{\textwidth}{|p{0.5in}|X|}
  \hline
  \centerline{\textbf{\large Time}} &  \textbf{\large Instructional Sequence } \\
  \hline

  \textbf{10 min} & \textbf{\em Introduction/Anticipatory Set:} 6
  terms are written on the board, and students are to simplify the
  expressions using the rules of exponents.  The students use their
  whiteboards to write their answers, and when they are finished, they
  stand up.
  
  When about 80 percent of the class has stood up, have them sit down
  and discuss the answers they got with their partner for a few
  minutes.

  Then review the answers with the class. \\
  \hline
  
  \textbf{N/A} &  \textbf{\em Demonstrate, Build, Apply Knowledge} \\
  \hline
  \textbf{10 min} &  \textbf{\em Depth of Knowledge Questions:} Previously assigned workbook questions are reviewed.  \\
  \\
  \hline
  \textbf{N/A} &  \textbf{\em Guided Practice:} N/A \\
  \hline
  \textbf{10 min} &  \textbf{\em Independent Practice:} A game called ``Quiz-Quiz Trade'' was played.  A 3x5 card is given to every student.  On one side, there is a question about the rules of exponents.  On the other side is the answer.  The game proceeds by repeating the following steps:
 
  {\begin{enumerate}
    \item Students stand up and go around the classroom, looking for somebody who they haven't quizzed yet.
   \item (The ``Quiz-Quiz'' step) The students take turns asking their
     partner the question on the card they have.  The partner gives and answer, and which is checked against the correct answer on the back of the card.
   \item (The ``Trade'' step)  Then the students exchange cards, and go find somebody they haven't done this yet with.
 \end{enumerate}} \\
 \hline
 
 \textbf{40 min} &  \textbf{\em Assessment:}  The first test of the semester was administered to them. \\
 \\
 \hline
 \textbf{N/A} &  \textbf{\em Wrap Up/Closing Activity/Reflection:} After finishing the test, the students were encouraged to take a little quiz which determines whether they are language, visual, or kinesthetic learner. \\
 \hline
\end{tabularx}

\vskip 6pt

\begin{small}
  \begin{tabularx}{\linewidth}{|p{2.1in}|X|}
    \hline
    \textbf{Differentiation/Accommodations and/or Modifications: } & One of our sessions is an inclusion class, and students who were not comfortable socially were allowed to opt out, while an Aide reviewed with them.\\
    \hline
    \textbf{Culturally Responsive Teaching/ Diversity and Inclusion: } & N/A \\
    \hline
  \end{tabularx}
  
\vskip 6pt

\begin{tabularx}{\linewidth}{|X|}
  \hline
  \textbf{Self-Reflection:} \\
  \textbf{My Teaching:} 
  {\begin{enumerate}
  \item I am wondering whether the ``Quiz-Quiz Trade'' questions were extensive enough to prep them for the test.
  \item it is a delicate balancing act: one one hand, we want enough questions to cover what needs to be reviewed.  On the other, we want as few questions as possible, so the students see them over and over again, giving them a robust review of the material.
  \end{enumerate}} \\
  
  \textbf{The Students:}
  {\begin{enumerate}
  \item 8th graders are so fun.  They loved the ``Quiz-Quiz Trade''
    game.  They enthusiastically played it with each other, and kept
    trading partners until we ran out of time. I don't think my
    upperclass math classes which I have taught would have reacted as
    enthusiastically.
  \end{enumerate} }\\
  
  \textbf{The Lesson:}
  {\begin{enumerate}

    \item The ``Quiz-Quiz Trade'' game is remarkably effective at
      getting kids to actually attempt to answer the questions and
      review the material.
      
    \item If we had just put the same questions on a worksheet, for
      example, 80 percent of the class would have started working on
      them, reluctantly, 10 percent would have worked on them after
      direct teacher interaction, and 2 or 3 would have not worked on
      them at all.  Yet they all went around asking each other
      questions and giving answers.  Astounding.

    \item We just set a timer to let us now when to stop playing the
      game, but perhaps we should introduce some more formal mechanism
      to track whether each kid is getting a chance to talk with every
      other kids.  Mrs. Toelle assures me that 8th-graders still love
      stickers, so perhaps we should give a sheet of stickers to every
      student, and then have them past them on a sheet, and when
      they've collected all of the stickers they know they are done.

    \item I'm wondering whether we should do a round of ``Quiz-Quiz
      Trade'' every day, or perhaps several times per day, after 10
      minutes of lecture telling them how to do the problems on the
      cards.  It gets kids up out of their seats, walking and
      oxygenated and therefore fully awake, and--importantly--doesn't
      let any student just opt-out of class by zoning out during the
      lectures and then refusing to do the workbook problems.

  \end{enumerate} }\\
  \hline
\end{tabularx}
\end{small}
